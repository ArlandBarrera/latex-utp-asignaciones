% ------------------------------
% Lista de temas y subtemas
% ------------------------------
% NOTE: Elementos en una lista
% referencia: previamente se define el directorio del tema
% Separar: es recomendable usar comas, no espacios
% Nombrar: NO USAR guion (-), barra baja (_) o espacio (' ')
% Definir: usar el mismo nombre del directorio del tema

% ----------- TEMAS -----------
% Temas de capitulos personalizados
% Aqui se colocan los temas definidos por el usuario
% Mirar el ejemplo abajo
% descomentar y editar la siguiente linea de abajo
% \newcommand{\temas}{}

% NOTE: Esto es solo un ejemplo
% Eliminar o comentar la linea de abajo y definir nuevos temas arriba
% Eliminar los direcotorios respectivos si no se requieren

\newcommand{\temas}{nuevotema, temauno, temados}

% ----------- SUBTEMAS -----------
% Subtemas especificos dentro de los capitulos
% Aqui se colocan los subtemas definidos por el usuario
% Mirar el ejemplo abajo
% descomentar y editar la siguiente linea de abajo
% \newcommand{\subtemas}{}

% NOTE: Esto es solo un ejemplo
% Eliminar o comentar la linea de abajo y definir nuevos subtemas arriba
% Eliminar los direcotorios respectivos si no se requieren

\newcommand{\subtemas}{subtemauno,subsubtemauno,profundouno,subtemados,subsubtemados,profundodos}

% ------------------------------
% Hipervinculos
% ------------------------------
% NOTE: Si no se necesitan hipervinculos
% Eliminar los links de temas y subtemas
% Comentar o eliminar la linea de abajo y/o eliminar el archivo indicado

\input{hipervinculos.tex}

% ------------------------------
% Automatizar [in]put a [cap]itulos -> \incap
% ------------------------------
\newcommand{\incap}[1]{
  \input{#1/ref-#1.tex}
}

% [It]erar temas con [in]put [cap]itulo -> \itincap
%
\newcommand{\itincap}[1]{
  \forcsvlist{\incap}{#1}
}

% ------------------------------
% Lista de graficas
% Nuevo entorno flotante para graficas
% ------------------------------

% ---------- Descomentar para habilitar ----------
% \newfloat{grafica}{htbp}{lop}[chapter]
% \floatname{grafica}{Gráfica}

% ------------------------------
% Lista de ecuaciones
% Nuevo entorno flotante para ecuaciones
% ------------------------------

% ---------- Descomentar para habilitar ----------
% \newfloat{ecuacion}{htbp}{loe}[chapter]
% \floatname{ecuacion}{Ecuación}

% Listar y crear caja (box) alrededor de una ecuacion
% \newcommand{\listequbox}[3]{
%   \begin{ecuacion}
%     \[
%       \ensuremath{\boxed{#1}}
%       \label{#2}
%     \]
%     \caption{#3}
%   \end{ecuacion}
% }

% ------------------------------
% Texto reutilizable - Strings
% ------------------------------

\input{strings.tex}
